Dieses Experiment wurde im Rahmen des Physikunterichts der Klasse 6MT13v der BBW durchgeführt. Begleitende Lehrperson war X. Würms.

Diese Arbeit zeigt auf wie der Luftwiederstand auf Objekte wirkt, in unserem Experiment auf die Auswirkung auf Papiertrichter.

Der Luftwiderstand ist eine Bezeichnung für eine Kraft, welche auftritt, wenn zum Beispiel ein Papiertrichter fällt. Ohne den Luftwiderstand wäre der Fall physikalisch gesehen ein Freier Fall und der Papiertrichter würde um 9.81 \(\frac{m}{s}\) pro Sekunde schneller werden. Das wäre eine gleichmässig beschleunigte Bewegung (die Geschwindigkeit nimmt linear mit der Zeit des Falles zu). 

Die Geschwindigkeit des Papiertrichters wäre innerhalb von kurzer sehr gross. Allerdings befindet sich Luft zwischen dem Papiertrichter und dem Boden. Diese bremst den Papiertrichter ab, bis er nicht mehr beschleunigt und er mit einer konstanten Geschwindigkeit fällt.
