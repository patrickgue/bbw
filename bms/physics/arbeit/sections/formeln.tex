\begin{multicols}{2}
	

\(v^2 \sim m \sim F_L\)

\subsection{Laminar}

\(F_L = k * v\)

\subsection{Turbulenzen}

\(F_T = k * v^2 \)

\subsection{\label{sec:widerst} Widerstandskoeffizient}

\( c_w = \frac{2F_w}{\rho v^2 A} \) \footnote{\cite{wikiwidkoef} Abschnitt \textit{Definition}}

\subsection{Widerstandskraft}

\(F_w = \frac{\rho c_w A v^2}{2} \) \footnote{Ebd. Abschnitt \textit{Anwendung}}

Ist die Endgeschwindigkeit erreicht ist, ist \(F_w = -F_g\) da keine weitere Beschleunigung auf den fallenden Körper mehr auswirkt.

\begin{figure}
	\centering
	\begin{tabular}{|l|l|l|}
		\hline
		\textbf{v} & \textbf{\(F\) Laminar} & \textbf{\(F\) Turbulent} \\
		\hline
		0 & 0 & 0\\ \hline
		1 & 1 & 1\\ \hline
		2 & 2 & 4\\ \hline
		3 & 3 & 9\\ \hline
		4 & 4 & 16\\ \hline
		5 & 5 & 25\\ \hline
		
	\end{tabular}
	\caption{\label{fig:tablegraph}}
\end{figure}

\end{multicols}
