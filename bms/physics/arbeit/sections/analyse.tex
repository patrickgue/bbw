Die Auswertung der Resultate zeigt mehrere Punkte. Es zeigt, dass die erreichte Geschwindigkeit zwar vom Gewicht des Testobjekts abhängt, diese aber nicht linear beeinflusst:

\begin{tabular}{rrrrr}
	\textbf{[n] Kegel} & \(v[\frac{m}{s}]\) & m[g] & {\small \(v\) in Rel. zu Exp. 1} & {\small \(m\) in Rel. zu Exp. 1} \\ \hline
	1 & -1.245 & 1.5 & \textit{100\%} & \textit{100\%} \\
	2 & -1.790 & 3   & 143\% & 200\% \\
	3 & -2.100 & 4.5 & 167\% & 300\% \\
	4 & -2.550 & 6   & 205\% & 400\% \\
	5 & -2.980 & 7.5 & 239\% & 500\% \\
\end{tabular}

\subsection{Widerstandskoeffizient}

Der Widerstandskoeffizient kann mit einer Formel in im Abschnitt \ref{sec:widerst} berechnet werden. Es werden folgende Parameter für die Berechnung verwendet:
\begin{itemize}
	\item \(F_w = F_g = m * g\)
	\item \(\rho = 1.2041 \frac{kg}{m^{3}} \)\ \textit{(Bei \(20^\circ C \) auf 0 m.Ü.M. \footnote{\cite{wikiluftdichte} Abschnitt: \textit{Einführung}})}
	\item \(A = 132.736 cm^2 = 0.0132736 m^2\) \textit{(Siehe \ref{flaeche})}
\end{itemize}

Berechnung von $ c_w $: \\

Formel: \\
$ c_w = \frac{2 * {m} * {g} }{\rho  * A * {v_E} ^{2}} $

1 Trichter: \\
 $ c_w = \frac{2 * 0.0015kg * 9.81\frac{m}{s^{2}}}{1.2041\frac{kg}{m^3}  * 0.0132 m^2 * (1.245\frac{m}{s})^2} = 1.187955785$

2 Trichter: \\
 $ c_w = \frac{2 * 0.003kg * 9.81\frac{m}{s^{2}}}{1.2041\frac{kg}{m^3}  * 0.0132 m^2 * (1.79\frac{m}{s})^2} = 1.149378088$

3 Trichter: \\
 $ c_w = \frac{2 * 0.0045kg * 9.81\frac{m}{s^{2}}}{1.2041\frac{kg}{m^3}  * 0.0132 m^2 * (2.1\frac{m}{s})^2} = 1.252626643$

4 Trichter: \\
 $ c_w = \frac{2 * 0.006kg * 9.81\frac{m}{s^{2}}}{1.2041\frac{kg}{m^3}  * 0.0132 m^2 * (2.55\frac{m}{s})^2} = 1.132709675$

5 Trichter: \\
 $ c_w = \frac{2 * 0.0075kg * 9.81\frac{m}{s^{2}}}{1.2041\frac{kg}{m^3}  * 0.0132 m^2 * (2.98\frac{m}{s})^2} = 1.036755758$

Mittelwert von $c_w$ aus unserem Experiment ist 1.15188519

