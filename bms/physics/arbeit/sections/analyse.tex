Die Auswertung der Resultate zeigt mehrere Punkte. Es zeigt, dass die erreichte Geschwindigkeit zwar vom Gewicht des Testobjekts abhängt, diese aber nicht linear beeinflusst:

\begin{tabular}{rrrrr}
	\textbf{[n] Kegel} & \(v[\frac{m}{s}]\) & m[g] & {\small \(v\) in Rel. zu Exp. 1} & {\small \(m\) in Rel. zu Exp. 1} \\ \hline
	1 & -1.245 & 1.5 & \textit{100\%} & \textit{100\%} \\
	2 & -1.790 & 3   & 143\% & 200\% \\
	3 & -2.100 & 4.5 & 167\% & 300\% \\
	4 & -2.550 & 6   & 205\% & 400\% \\
	5 & -2.980 & 7.5 & 239\% & 500\% \\
\end{tabular}

\subsection{Widerstandskoeffizient}

Der Widerstandskoeffizient kann mit einer Formel in im Abschnitt \ref{sec:widerst} berechnet werden.