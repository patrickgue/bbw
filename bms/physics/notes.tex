% Created 2016-11-07 Mo. 09:41
\documentclass[11pt]{article}
\usepackage[latin1]{inputenc}
\usepackage[T1]{fontenc}
\usepackage{fixltx2e}
\usepackage{graphicx}
\usepackage{longtable}
\usepackage{float}
\usepackage{wrapfig}
\usepackage{rotating}
\usepackage[normalem]{ulem}
\usepackage{amsmath}
\usepackage{textcomp}
\usepackage{marvosym}
\usepackage{wasysym}
\usepackage{amssymb}
\usepackage{hyperref}
\tolerance=1000
\usepackage[margin=1in]{geometry}
\date{\today}
\title{Notizen}
\hypersetup{
  pdfkeywords={},
  pdfsubject={},
  pdfcreator={Emacs 25.1.1 (Org mode 8.2.10)}}
\begin{document}

\maketitle
\tableofcontents


\section{242)}
\label{sec-1}

\textbf{Skizze}

\begin{verbatim}
  F   /
  /  /
 o  /
Fab/
  /
 /
/(a)
---------------
\end{verbatim}

\textbf{Gegeben}
\(\alpha = arctan(0.05) \)

\(P = 1 kW\)

\(m = 1000 kg\)


\textbf{Formeln}

\(P = F * v\)

\textbf{Berechnungen}

\(F = F_{ab}\)


\(F_{ab} = F_{G,P} + F_R\)


\(V = \frac{P}{F} = \frac{P}{F_{G,P} + F_R } = \frac{P}{F_G * sin(\alpha) + \mu * F_G cos(\alpha)}
 = \frac{1000 W}{1000 kg * 9.81\frac{m}{s^2}  (sin(arctan(0.05) + 0.06 * cos(arctan(0.05)))  } = 0.9 \frac{m}{s} \) 


\section{Notizen zum Wirkungsgrad}
\label{sec-2}

\emph{Wirkungsgrad} ist: Abgegebene Leistung geteilt durch aufgenommene Leistung

\(\eta = \frac{P_{ab}}{P_{zu}} = \frac{W_{ab}}{t} * \frac{t}{W_{zu}} = \frac{W_{ab}}{W_{zu}} = \frac{\epsilon_{ab}}{\epsilon_{zu}} < 1\)



eta = Nutzen / Aufwand



\textbf{Gesamtwirkungsgrad}

\(P_{ab} = \eta_1 * ... * \eta_n * P_{zu}\)



\section{244)}
\label{sec-3}

\(\eta = \frac{P_{ab}}{P_{zu}}\)

\section{Kinetische Energie}
\label{sec-4}

\(E_k = \frac{1}{2}mv^2\)

\section{Reibungsarbeit}
\label{sec-5}

\(F_R * s\)

W�rme(energie), innere Energie

\section{Andere Energieformen}
\label{sec-6}

\begin{itemize}
\item W�rme
\item chemische
\item Kernenergie
\end{itemize}

\section{Energieenrhaltung}
\label{sec-7}
Energie kann nur die Form �ndern. Sie kann wieder erzeugt, noch vernichtet werden. \(E_{tot} = konst.\)

\section{\underline{Experiment} fadenpendel}
\label{sec-8}
Vergleich von \(V_{max}\) aus Theoriemit Experiment


\section{245)}
\label{sec-9}
\begin{verbatim}
       v_0 /|\
     o
 ____| 
 |       o
_|_______|______
        \|/ v_x
\end{verbatim}

\(h = 3m\)


\(E_{tot,1} = E_{tot,2}\)

Vorhandene Energieen
\begin{itemize}
\item \(E_{kin,1} + E_{pot,1} = E_{kin,2} \)
\item \( \frac{1}{2}mv_0^2 + mgh = \frac{1}{2}mv_x^2 | /m | * 2  \)
\item \(v_x = \sqrt{v_0^2 + 2gh}\)
\end{itemize}

\section{250)}
\label{sec-10}
\(m = 10g\)

\(h = 0.9m\)
% Emacs 25.1.1 (Org mode 8.2.10)
\end{document}
