\documentclass{article}
\usepackage[utf8]{inputenc}
\usepackage[margin=.4in]{geometry}
\usepackage{amsmath}
\usepackage[parfill]{parskip}
\usepackage{multicol}
\newcommand{\inv}{^{\text{-}1}}

\renewcommand{\familydefault}{phv}

\begin{document}
	\begin{multicols}{3}
		\section{Logarithmus Funktionen}
		\subsection{Grundsätzliche Dinge}
		Zehnerlogarithmus: \(lg\)\\
		Natürlicher Logarithmus: \(ln \equiv log_e \)\\
		\(e^{log_{e}(a)} = a \)\\
		\subsection{Grundfunktionen}
		\(log_b(xy) = log_b(x) + log_b(y)\)\\
		\(log_b(\frac{x}{y}) = log_b(x) - log_b(y)\)\\
		\(log_b(x^p) = p * log_b(x) \)\\
		\(log_b(\sqrt[p]{x}) = \frac{log_b(x)}{p} \)\\
		\(log_b(x) = \frac{log_k(x)}{log_k(b)} \)
	
		\subsection{Umkehrfunktion}
		Schreibweise: \(f\inv \) 
		
		\section{Vektoren}
		\subsection{Schreibweise}
		\(\vec{a} = \begin{bmatrix}
		a_1\\a_2\\a_3\end{bmatrix} \)
		\subsection{Operationen}
		\(\vec{a} = \begin{bmatrix}a_1\\a_2\\a3\end{bmatrix},
		\vec{b} = \begin{bmatrix}b_1\\b_2\\b3\end{bmatrix}\)\\
		\textbf{Addition:}\\
		\(\vec{a} + \vec{b} = \begin{bmatrix}
		a_1 + b_1\\a_2 + b_2\\a_3 + b_3\\
		\end{bmatrix} \)	
	\end{multicols}
	
	
\end{document}