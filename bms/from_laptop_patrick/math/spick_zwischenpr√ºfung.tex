\documentclass[a4]{article}
\usepackage[utf8]{inputenc}
\usepackage{utopia}
\usepackage{graphicx}
\usepackage[top=1in, bottom=1in, left=1in, right=1in]{geometry}

\title{Schule\\
	Spick zwischenprüfung
}
\author{Patrick Günthard}
\date{\today}
\DeclareGraphicsExtensions{.pdf,.png,.jpg}

\begin{document}
	\maketitle

\section{quadratische Gleichungen}
\label{sec-1}

\subsection{Mitternachtsformel}
\label{sec-1-1}

Basic formula:
\begin{equation}
\frac{-b \pm \sqrt[]{b^2 - 4ac}}{2c}
\end{equation}

alternative formula:
\begin{equation}
x^2 + px + q = 0
\end{equation}
\begin{equation}
x_1, x_2 = - \frac{p}{2} \pm \sqrt[]{ (\frac{p}{2})^2 - q }
\end{equation}

\subsection{scheitelpunkt:}
\label{sec-1-2}

\subsubsection{Wenn a und b bekannt:}
\label{sec-1-2-1}

\begin{equation}
\frac{-b}{2a}
\end{equation}

\subsubsection{Wenn a oder b unbekannt}
\label{sec-1-2-2}
Basis: Mitternachtsformel
\begin{equation}
\frac{-b \pm \sqrt[]{b^2 - 4ac}}{2a}
\end{equation}

Formel Scheitelpunkt, die \emph{Diskriminante}:

\begin{equation}
b^2 - 4ac = 0
\end{equation}
Diese Gleichung hat nur eine lösung

\section{Brüche}
\label{sec-2}
\subsection{Mehrfachbrüche}
\label{sec-2-1}

\begin{equation}
\frac{\frac{a}{b}}{\frac{c}{d}} = \frac{a}{b} * \frac{d}{c}
\end{equation}

\section{Substitution}
\label{sec-3}

Beispiel:
\begin{equation}
2*cos(x)^2 + cos(x)
\end{equation}
\begin{equation}
u = cos(x)
\end{equation}
\begin{equation}
2u^2 + u = 0
\end{equation}
\begin{equation}
u_1, u_2 = \frac{-2 \pm \sqrt[]{1 - 0}}{4}
\end{equation}
\begin{equation}
u_1 = -\frac{3}{4}\\
u_2 = -\frac{1}{4}
\end{equation} 
\end{document}