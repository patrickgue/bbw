\documentclass[12pt,twoside,titlepage,a4paper]{article}
\usepackage[utf8]{inputenc}
\usepackage[english]{babel}
\usepackage[margin=.6in]{geometry}
\usepackage[parfill]{parskip}

\usepackage[onehalfspacing]{setspace}
\usepackage{fancyhdr}
\usepackage{lastpage}
\usepackage{hyperref}
\renewcommand{\textsf}{phv}
\usepackage[sf,bf]{titlesec}


%\titleformat*{\section}{\LARGE\bfseries\textsf}



\title{Summary and Notes: \textit{The Picture of Dorian Gray} by \textit{Oscar Wilde}}
\author{Patrick Günthard}

\pagestyle{fancy}
\fancyhf{}
%\lhead{}

\fancyhead[EL]{Summary and Notes: The Picture of Dorian Gray}
\fancyhead[OR]{Patrick G\"unthard}
%\rhead{Patrick Günthard}
\cfoot{\thepage \space von \pageref{LastPage}}


\begin{document}
	\maketitle
	\tableofcontents
	
	\section{Summary}
	
	
	
	Dorian Gray meets Lord Henry Wotton at the studio of Basil Hallward, who is using Dorian as a model for his latest painting. Lord Henry tells Dorian about his epicurean views on life, and convinces him of the value of beauty above all other things. The young and impressionable Dorian is greatly moved by Lord Henry's words. When Basil shows them the newly completed painting, Dorian is flooded with awe at the sight of his own image, and is overwhelmed by his fear that his youth and beauty will fade. He becomes jealous that the picture will be beautiful forever while he is destined to wither and age. He passionately wishes that it could be the other way around. Lord Henry is fascinated with Dorian's innocence as much as Dorian is impressed by Henry's cynically sensual outlook on life. They become fast friends, to Basil's dismay. He fears that Henry will be a corrupting influence on the young, innocent Dorian, whom he adores.
	
	Dorian and Lord Henry become fast friends, often dining together and attending the same social functions. Henry's influence has a profound effect on the young man, who soon adopts Henry's views as his own, abandoning ethical restraints and seeing life in terms of pleasure and sensuality. Dorian falls in love with the beautiful Sibyl Vane, a poor but talented young Shakespearean actress. They are engaged to be married until Dorian brings Henry and Basil to a performance, where her acting is uncharacteristically - and inexplicably - terrible. Dorian confronts Sibyl backstage, and she tells him that since she is now truly in love, she no longer believes in acting. Disgusted and offended, Dorian breaks off their engagement and leaves her sobbing on the floor. When he returns home, he discovers that the figure in his portrait now bears a slightly different, more contemptuous facial expression.
	
	Dorian awakens late the next day feeling guilty for his treatment of Sibyl, and writes an impassioned love letter begging her forgiveness. Soon, however, Lord Henry arrives, and informs Dorian that Sibyl committed suicide last night. Dorian is shocked and wracked with guilt, but Henry convinces him to view the event artistically, saying that the superb melodrama of her death is a thing to be admired. Succumbing to the older man's suggestion, Dorian decides that he need not feel guilty, especially since his enchanted portrait will now bear his guilt for him. The picture will serve as his conscience, allowing him to live freely. When Basil visits Dorian to console him, he is appalled at his friend's apathy towards Sibyl's death. Dorian is unapologetic and annoyed by Basil's adulation of him.
	
	Paranoid that someone might discover the secret of the painting, and therefore the true nature of his soul, Dorian hides the image in his attic. Over the next several years, Dorian's face remains young and innocent, despite his many selfish affairs and scandals. He is an extremely popular socialite, admired for his fine taste and revered as a fashionable trend-setter. The picture, however, continues to age, and grows more unattractive with each foul deed. Dorian cannot keep himself from looking at the picture periodically, but he is appalled by it, and is only truly happy when he manages to forget its existence. He immerses himself in various obsessions, studying mysticism, jewelry, music, and ancient tapestries. These interests, however, are all merely distractions that allow him to forget the hideousness of his true soul.
	
	One night, Basil visits Dorian to confront him about all of the terrible rumors he has heard. The painter wants to believe that his friend is stll a good person. Dorian decides to show him the portrait so that he can see the true degradation of his soul, but when Basil sees it he is horrified, and urges his friend to repent for his sins. Basil's reaction enrages Dorian, and he murders the artist with a knife. To dispose of the body, he blackmails an estranged acquaintance, Alan Campbell, a chemist who is able to burn the body in the attic's fireplace. Alan has already been driven into isolation by Dorian's corrupting influence, and this action eventually compels him to commit suicide.
	
	Not long after, Dorian visits an opium den and is attacked by James Vane, Sibyl's brother, who has sworn revenge on the man that drove his sister to suicide. 18 years have passed since the event, however, yet Dorian still looks like a 20-year-old youth. James thinks that he is mistaken, and Dorian escapes before his would-be murderer learns the truth. Over the next several days Dorian lives in fear, sure that James is searching for him. While hunting one day, Dorian's friend Geoffrey accidentally shoots a man hiding on Dorian's property. This stranger is revealed to be James Vane. Dorian is overcome with relief, but cannot escape the fact that four deaths now weigh on his conscience.
	
	Deciding to change his life for the better, Dorian commits a good deed by refusing to corrupt a young girl who has fallen in love with him. He checks the portrait, hoping to find that it has changed for the better, but when he realizes that the only thing that has changed is the new, hypocritical smirk on the wrinkled face, he realizes that even his effort to save his soul was driven by vanity. In a fit of despair, he decides to destroy the picture with the same knife that he used to kill Basil, its creator. Downstairs, Dorian's servants hear a shriek, and rush upstairs to find their master dead on the floor, the knife plunged into his own chest. Dorian's youthful countenance is gone, and his servants are only able to recognize him by the jewelry on his fingers.
	
	\section{Characters}
	
	\paragraph{Basil Hallward} A reclusive painter much respected by the London aristocracy. He admires Dorian to the point of adulation and paints many portraits of him, finally creating his masterpiece, the titular picture. Basil introduces Dorian to Lord Henry Wotton.
	
	
	\paragraph{Lord Henry Wotton} A champion of sensual pleasure, notorious among London's high society for his dazzling conversation and brazenly immoral views. He values beauty above all else, and is chiefly responsible for Dorian's corruption.
	
	\paragraph{Dorian Gray} A physically beautiful young man, naive and good-hearted until corrupted by vanity. Dorian makes a faustian bargain: his body remains young and beautiful, while his portrait alters to reflect his age and increasingly guilty conscience. He eventually seems to bring corruption, pain, and death to all inhabitants of the social circles in which he moves. 
	
	\paragraph{Lord George Fermor} Henry Wotton's uncle, an idle, impatient aristocrat. Henry calls on him to elicit information about Dorian's background. He is a portrait of a typical self-centered, elderly aristocrat whose money allows him to devote his life to purely fanciful and superficial endeavors.
	
	\paragraph{Sibyl Vane} A beautiful, 17-year-old Shakespearean actress, and Dorian's first love. The pair are smitten with each other and are engaged to be married until Dorian sees her perform badly, and, disillusioned, treats her with extreme cruelty. Broken-hearted, she commits suicide.
	
	\paragraph{Mrs Vane} Sybil's aging, single mother. Mrs Vane is also an actress, and both she and her daughter struggle to support their small family through their craft. She is most comfortable when her real life is as melodramatic as it is on the stage.
	
	\paragraph{James Vane} Sibyl's younger, fiercely protective brother, who leaves England to become a sailor. He is suspicious of his sister's lover from the start, and swears to hunt the man down if he causes her any harm. After Sibyl's death, he dedicates himself to finding his sister's "Prince Charming", and is eventually killed by a wayward hunting bullet while trying to take his revenge on Dorian.
	
	\paragraph{Mr Isaacs} The man who runs the decrepit theater where Sybil performs. The Vanes are deeply in debt to him. He is a sterotypical portrait of an old Jewish man, whom Dorian and Basil find contemptible, and whom Lord Henry finds amusing.
	
	\paragraph{Victor} Dorian's faithful first servant, of whom he is unnecessarily suspicious. Victor has been replaced by another servant by the second half of the novel, although the details of his dismissal are never disclosed. We are left to surmise that either Dorian's paranoia became too great, or that Victor eventually grew unable to bear his master's increasingly corrupt nature.
	
	\paragraph{Mr Hubbard}A celebrated London frame-maker whom Dorian calls upon to help him hide the portrait in the attic. He appears only once in the novel, but stokes Dorian's growing paranoia by being puzzled when the protagonist adamantly refuses to uncover the painting for him to see it.
	
	\paragraph{Adrian Singleton} A promising young member of society whose life takes a turn for the worse when he befriends Dorian. Adrian ends up addicted to opium, spending all of his time and money in filthy, dilapidated drug dens.
	
	\paragraph{Alan Campbell} A talented chemist and musician who is close to Dorian until their friendship comes to a bitter end as a consequence of Dorian's increasingly bad reputation. Dorian forces him to assist in the disposal of Basil's body using blackmail, and Alan later commits suicide.
	
	\paragraph{Lady Narborough} The widow of a wealthy man, and the mother of richly married daughters. She hosts a great many parties, and is very fond of Dorian and Lord Henry.
	
	\paragraph{Sir Geoffrey Clouston} A London socialite and guest of Lady Narborough who shoots James Vane in a hunting accident. Unlike most of the aristocrats present at the incident, he appears to be quite disturbed by the idea of having taken a human life.
	
	\paragraph{Lady Alice Chapman} Lady Narborough's decidedly unremarkable daughter, a minor character whom Wilde uses to display Lord Henry's superficiality.
	
	\paragraph{Duchess of Monmouth} Gladys, a clever and pretty young aristocrat who nearly matches Lord Henry in conversational wit. She freely and lightly admits to numerous adulterous affairs, and flirts with Dorian at one of his parties.
	
	\paragraph{Hetty Merton} A beautiful young village girl who falls in love with Dorian and reminds him of Sybil Vane. Dorian consciously - and hypocritically - refrains from corrupting her in an attempt to begin living a good life, and to purify his soul. She does not believe Dorian when he tells her that he is wicked, because he looks so young and innocent. She is the last young woman with whom Dorian is romantically linked.
	
	
\end{document}