% Created 2016-01-12 Di. 14:51
\documentclass[11pt]{article}
\usepackage[utf8]{inputenc}
\usepackage[T1]{fontenc}
\usepackage{fixltx2e}
\usepackage{graphicx}
\usepackage{longtable}
\usepackage{float}
\usepackage{wrapfig}
\usepackage{rotating}
\usepackage[normalem]{ulem}
\usepackage{amsmath}
\usepackage{textcomp}
\usepackage{marvosym}
\usepackage{wasysym}
\usepackage{amssymb}
\usepackage{hyperref}
\tolerance=1000
\usepackage{utopia}
\usepackage[left=1in; top=1in; right=1in; bottom=1in;]{geometry}
\date{\today}
\title{main}
\hypersetup{
  pdfkeywords={},
  pdfsubject={},
  pdfcreator={Emacs 24.5.1 (Org mode 8.2.10)}}
\begin{document}

\maketitle
\tableofcontents

\section{Schule 5. Semester}
\label{sec-1}
\subsection{Hausaufgaben}
\label{sec-1-1}
\subsubsection{Prüfungstermine}
\label{sec-1-1-1}
\begin{itemize}
\item Physik
\begin{itemize}
\item Termine 
\begin{itemize}
\item 29. 09. 2015
\begin{itemize}
\item Stoff: Skript S. 1 - 13:
\item gleichförmige Bewegung
\item gleichmässige beschleunigung
\item DPK ZA Blatt Kinematik 1 + 2
\item Häsli/Humm ZA Kinematik 1
\item 1 A4 Spick
\end{itemize}
\item 17. 11. 2015
\item 05. 01. 2015
\end{itemize}
\item Hilfsmittel
\begin{itemize}
\item TR ohne Grafik
\item 1 A4 Blatt Spick
\item Fundamentum
\end{itemize}
\end{itemize}
\item Mathematik
\begin{itemize}
\item 19. 09. 2015 Zwischenprüfung
\begin{itemize}
\item Bis zu diesem Zeitpunkt: Repetition
\end{itemize}
\item 17. 11. 2015 2. Prüfung
\item 19. 01. 2015 1. Prüfung für nächstes Semester
\end{itemize}
\item Deutsch
\begin{itemize}
\item 08. 09. 2015 Buchvorstellung
\item 22. 09. 2015 Aufsatz zu \emph{Am Hang - Markus Werner}
\item 15. 12. 2015 Hausaufsatz
\end{itemize}
\end{itemize}
\subsubsection{2015-08-18}
\label{sec-1-1-2}
\begin{itemize}
\item Projekt
\begin{itemize}
\item Gruppen bilden
\begin{itemize}
\item gleiche intressen
\end{itemize}
\item Ideen sammeln
\end{itemize}
\item Geschichte
\begin{itemize}
\item Google: Viktorianisches Zeitalter
\item Dossier Wettlauf um Afria S275 s280
\begin{itemize}
\item Zeitstrahl
\end{itemize}
\end{itemize}
\item Mathematik
\item Deutsch
\begin{itemize}
\item Kapitel 1
\end{itemize}
\end{itemize}
\subsubsection{2015-08-25}
\label{sec-1-1-3}
\begin{enumerate}
\item Physik
\label{sec-1-1-3-1}
\begin{itemize}
\item Aufgaben 1,2,4,5,6
\end{itemize}
\item Deutsch
\label{sec-1-1-3-2}
\begin{itemize}
\item Charakterisierung um Loos und Clarin (\emph{Am Hang} bis S. 50) 
\begin{itemize}
\item 1x A4 12pt
\item 3 Zitate auf Satzebene
\item 3 Zitate auf Wortebene
\item 2 Zitate mit Doppelpunkt
\end{itemize}
\item Mailen bis 30. August 2015 23:59
\end{itemize}
\end{enumerate}
\subsubsection{2015-09-01}
\label{sec-1-1-4}
\begin{enumerate}
\item Physik
\label{sec-1-1-4-1}
\begin{itemize}
\item Aufgabenblatt (1. Aufgabe)
\end{itemize}
\item Deutsch
\label{sec-1-1-4-2}
\begin{itemize}
\item Buch vorstellen
\begin{itemize}
\item Zusammenfassung
\item Material zu diesem Buch?
\item Subjektive einschätzung
\end{itemize}
\end{itemize}
\end{enumerate}
\subsubsection{2015-11-10}
\label{sec-1-1-5}
\begin{enumerate}
\item Physik
\label{sec-1-1-5-1}
\begin{itemize}
\item Skript S. 14 - 16
\item ZA II, Nr.19
\item ZA III vollständig
\item HH Kinematik 6 (Diagramme, S27-30)
\item HH Kinematik 2 (Treffpunktaufgaben ohne Nr. 12)
\end{itemize}
\end{enumerate}
\subsection{Notizen}
\label{sec-1-2}
\subsubsection{2015-08-19}
\label{sec-1-2-1}
\begin{enumerate}
\item Mathe
\label{sec-1-2-1-1}
Aufgabe 1)
u = 2.0m
u = (pi)*r + 2r + 2xm
2.0m = (pi)*r + 2r + 2xm
???

Aufgabe 2)
   1

\rule{\linewidth}{0.5pt}
1 + tan$^{\text{2}}$(a)


1

\rule{\linewidth}{0.5pt}
cos$^{\text{2}}$(a)

\item Deutsch
\label{sec-1-2-1-2}
\textbf{Metatext}
meta|text

meta|kognition
----+---------
Über|Wissen

Metatext => "Text über den Inhalt des Textes"
\end{enumerate}
\subsubsection{2015-08-25}
\label{sec-1-2-2}
\begin{enumerate}
\item Physik
\label{sec-1-2-2-1}

\begin{verbatim}
    s2 - s1
v = -------
    t2 - t1
\end{verbatim}

[delta] = Delta
[delta]x = x2 - x1

\emph{falls s1 = 0 dann schreibt man nur s statt s2, gleiches bei t}


Umrechnung 1 km/h = 1000m / 3600s = 1/3.6 m/s (= 0.2777 m/s)


Frage: Wie kann eine negative Geschwindigkeit im s-t Diagramm erkannt werden?

\emph{Die Steigung ist negativ} 


\begin{enumerate}
\item Die Geschwindigkeit im s-t-Diagramm
\label{sec-1-2-2-1-1}


\begin{verbatim}
    [delta]y   y2 - y1
a = -- = -------
    [delta]x   x2 - x1
\end{verbatim}

Im Orts-Zeit Diagramm ist die Steigung daher

\begin{verbatim}
    [delta]s
m = -- = v
    [delta]t
\end{verbatim}

Zurückgelegter Weg
A = [delta]s


[delta]s = A = 1/2 [delta]v * [delta]t
\begin{verbatim}
 v2-------x
         /|
       /  |
[delta]v   /    |
   /  A   |
 /        |
0----------
   [delta]t
\end{verbatim}


\textbf{Weg-Zeit Funktion}

f(x) = ax + b -> s(t) = vt + s0

\begin{verbatim}
s |      __--
  |  __--
s0|--
  |
  |
  ---------------------
\end{verbatim}
s(t) = v * (t - t0) + s0

t0 = startzeit

\item Aufgaben
\label{sec-1-2-2-1-2}

\begin{itemize}
\item Aufgabe 1
\begin{itemize}
\item schätzen
\begin{enumerate}
\item ca. 15m
\item 0.001s
\item 0.000001s
\item 300m
\item 10cm
\item 15km/h
\end{enumerate}
\item berechnen
\begin{enumerate}
\item 60 km/h -> 60000m / 3600s -> \textbf{16m/s}
\item 340m/s -> 34000m / 1s -> 10m / 34000m * 1s = 0.00029412s
\item Lichtgeschwindigkeit im vaakum: 29'979'200'000 cm/s, 10 / lightspeed = 3.3356460479265624e-010 = 0.00000000033357s
\item Lichtgeschwindigkeit im vaakum: 29'979'200'000 cm/s, * 1/1000s = 29979200cm = 299792m
\item 110 km/h -> 110000m / 3600s -> 0.061m = 6.1cm
\item (/ (/ 100  13.2) (/ 1000.0 3600.0)) = 27.273
\end{enumerate}
\end{itemize}
\item Aufgabe 2
\begin{itemize}
\item 1.6s * 1400m = 2240m; 2240 / 2 = \textbf{1120m}
\end{itemize}
\item Aufgabe 4
\end{itemize}
\end{enumerate}
\end{enumerate}

\subsubsection{2015-09-01}
\label{sec-1-2-3}
\begin{enumerate}
\item Projekt
\label{sec-1-2-3-1}
\begin{itemize}
\item Film über Gentrifikation (e.g. Kreuzberg)
\end{itemize}
\item Physik
\label{sec-1-2-3-2}
\begin{equation}
60km/h * t = -80km/h * t + 430km | + 80km * t
\end{equation}
\begin{equation}
140km/h * t = 430km | 140km/h
\end{equation}
\begin{equation}
t = 340km/140km/h
\end{equation}
\begin{equation}
t = \frac{34}{14}h = 3.07h
\end{equation}
\item Deutsch
\label{sec-1-2-3-3}
\begin{itemize}
\item 
\end{itemize}
\end{enumerate}
\subsubsection{2015-09-08}
\label{sec-1-2-4}
\begin{enumerate}
\item English
\label{sec-1-2-4-1}
\begin{itemize}
\item Test on 2015-09-22
\begin{itemize}
\item present perfect simple vs present perfect continious
\item Collotations
\end{itemize}
\end{itemize}
\end{enumerate}

\subsubsection{2015-09-15}
\label{sec-1-2-5}
\begin{enumerate}
\item Physik
\label{sec-1-2-5-1}
\begin{itemize}
\item Repetition Treffpunktaufgaben
\end{itemize}
J: 09:00
G: 11:00

v$_{\text{j}}$: Mach 0.9 = 297m/s
v$_{\text{g}}$: Mach 2.4 = 813.6m/s

t$_{\text{d}}$ (Zeitdifferenz) = 2h

s$_{\text{j}}$ = v$_{\text{j}}$ * t
s$_{\text{g}}$ = v$_{\text{g}}$ * t - t$_{\text{d}}$


2928.96km/h * t[h] = 1069.2km/h * (t[h] - 2h)


\begin{equation}
t = \frac{-v_g * t_d}{v_j - v_g} 
\end{equation} 
\item English
\label{sec-1-2-5-2}
\end{enumerate}

\subsubsection{2015-09-22}
\label{sec-1-2-6}
\begin{enumerate}
\item Physik
\label{sec-1-2-6-1}

Beschleunigung:
\begin{equation}
a = \frac{d_v}{d_t}
\end{equation}

\begin{equation}
[a] = \frac{\frac{m}{s}}{s} = \frac{m}{s^2} = m * s^{-2}
\end{equation}

\begin{equation}
v = a * t
\end{equation}

Am Beispiel eines Autos; "von 0 auf 100 in 15.4s"

\begin{equation}
a = \frac{\frac{100m}{3.6s} - 0\frac{m}{s}}{15.4s} = 1.80 \frac{m}{s^2}
\end{equation}

Der zurückgelegte Weg entspricht der Fläche des folgenden Diagramms

\begin{verbatim}
km/h
|      /|- lineare beschleunigung
      / |
_____/__|/konstante geschwindigkeit
|   /   |
|  /    |
| /     |
|/______| t
\end{verbatim}

Geschwindigkeit eliminieren
\begin{equation}
v = \frac{1}{2} at^2
\end{equation}

Strecke eliminieren
\begin{equation}
s = \frac{v^2}{2a}
\end{equation}

Wichtige Formeln:
\begin{equation}
s = \frac{1}{2}vt
\end{equation}

\begin{equation}
v = a * t
\end{equation}


\textbf{Aufgaben}

Aufgabe 9


\begin{equation}
g = 9.81\frac{m}{s}
\end{equation}


\begin{equation}
a_{max} = 10g, v = 7900 \frac{m}{s}
\end{equation}

\begin{equation}
s = \frac{v^2}{2a} = \frac{(7900\frac{m}{s})^2}{2 * 10 * 9,81\frac{m}{s^2}} = 318km
\end{equation}

Aufgabe 13)

\begin{enumerate}
\item 
\item 
\end{enumerate}
\begin{equation}
s = \frac{v^2}{2a} = \frac{(360km/h)^2}{(2*3.9m/s)^2} = \frac{10000m/s}{7.8m/s^2} = 1282.05m
\end{equation}

eine 1.2km lange Piste ist also 82.05m zu kurz

\begin{enumerate}
\item 
\end{enumerate}
\begin{equation}
v = \frac{\sqrt[]{as}}{2}
\end{equation}


Aufgabe 16)

\begin{equation}
a = 2.5\frac{m}{s^2}
\end{equation}




\item Mathe
\label{sec-1-2-6-2}
\begin{enumerate}
\item Lineare Gleichungen mit 2 Unbekannten
\label{sec-1-2-6-2-1}


\begin{equation}
ax + by = x | -ax
\end{equation}

\begin{equation}
by = -ax + c | / b
\end{equation}

\begin{equation}
y = - \frac{a}{b}x + \frac{c}{b}
\end{equation}

\textbf{Aufgaben}

\emph{380b}


\begin{equation}
2x + 5y = 30 | -2x
\end{equation}

\begin{equation}
5y = 30 -2x | / 5
\end{equation}

\begin{equation}
y = -\frac{2}{5}x + 30
\end{equation}

\emph{182a}

\begin{equation}
3x + y = 73
\end{equation}
\begin{equation}
2x -y = 32 | +
\end{equation}
\begin{equation}
5x = 105
\end{equation}
\begin{equation}
x = 21
\end{equation}
\begin{equation}
63 + y = 73 | - 63
\end{equation}
\begin{equation}
y = 10
\end{equation}
\emph{182d}

\begin{equation}
3a - 7b = 2 | *-5
\end{equation}
\begin{equation}
15a - 35b = 12
\end{equation}

\begin{equation}
-15a - 35b = -10 
\end{equation}
\begin{equation}
15a - 35b = 12
\end{equation}
\end{enumerate}
\end{enumerate}
\subsubsection{2015-09-29}
\label{sec-1-2-7}
\begin{enumerate}
\item Mathematik
\label{sec-1-2-7-1}
Aufgabe 389a)

Gleichung:


\begin{equation}
\frac{3}{f} + \frac{8}{g} = 3
\end{equation}

\begin{equation}
\frac{15}{f} - \frac{4}{g} = 4
\end{equation}

Substitution:
\begin{equation}
\frac{3}{f}
\end{equation}

\begin{equation}
\frac{4}{g} = 4
\end{equation}






\begin{equation}
x + 2 y = 3
\end{equation}

\begin{equation}
5x - y = 4 | * 2
\end{equation}

y fällt weg
 --
\begin{equation}
x = 1
\end{equation}

\begin{equation}
5 -4 = y
\end{equation}

\begin{equation}
y = 1
\end{equation}

\textbf{Lösung:}

\begin{equation}
\frac{3}{f} = 1; f = 3
\end{equation}

\begin{equation}
\frac{4}{g}; g = 4
\end{equation}


Aufgabe 389d)

Substitution:

\begin{equation}
u = \frac{1}{s}
\end{equation}
\begin{equation}
v = \frac{1}{t}
\end{equation}

--

Aufgabe 392a)

\begin{equation}
2x - 3x = -5u
\end{equation}
\begin{equation}
3x - 2y = -5v
\end{equation}


\item Deutsch
\label{sec-1-2-7-2}
\begin{enumerate}
\item Stil
\label{sec-1-2-7-2-1}
\end{enumerate}
\end{enumerate}

\subsubsection{2015-10-20}
\label{sec-1-2-8}
\begin{enumerate}
\item Physik
\label{sec-1-2-8-1}

\textbf{Aufgabe 26}

geg:
\begin{equation}
v_0 = \frac{90}{3.6}\frac{m}{s}
\end{equation}
\begin{equation}
t = 2.5s
\end{equation}
\begin{equation}
s_0 = 0m
\end{equation}
ges
\begin{equation}
a
\end{equation}

formulas:
\begin{equation}
s = v_0 t + \frac{1}{2}at^2 | -v_0 * t 
\end{equation}
\begin{equation}
s-v_0t = \frac{1}{2}at^2 | *\frac{2}{t^2}
\end{equation}
\begin{equation}
\frac{2(s-v_0t)}{t^2} = a
\end{equation}
\begin{equation}
\frac{2(55m - \frac{90m}{3.6s}*2.5s)}{(2.5s)^2}
\end{equation}

\textbf{Aufgabe 21}

geg.
\begin{equation}
v = 50\frac{km}{h}
\end{equation}
\begin{equation}
a = -6.8\frac{m}{s^2}
\end{equation}
\begin{equation}
s = 14m
\end{equation}

ges:

\begin{equation}
v_0
\end{equation}

\item Mathematik
\label{sec-1-2-8-2}
\end{enumerate}





\subsubsection{2015-10-27}
\label{sec-1-2-9}
\begin{enumerate}
\item Physik
\label{sec-1-2-9-1}

\textbf{Aufgabe 29} 

\begin{equation}
s_{tot} = s_R + s_B
\end{equation}
\begin{equation}
s_R = t_R * v_0
\end{equation}
\begin{equation}
s_B = \frac{v^2}{2a}
\end{equation}


\textbf{Aufgabe 33}

s = Strecke Zürich Bern

v-t Diagram:


\begin{verbatim}
  v[km/h]           __126
  | \              /
  |  \            /
  |   \          /
18|    \________/
  |____|________|____________t 
         1.3 km
\end{verbatim}

\begin{equation}
v_0 = \frac{126m}{3.6s} = 35\frac{m}{s}
\end{equation}
\begin{equation}
v_1 = \frac{18m}{3.6s} = 5\frac{m}{s}
\end{equation}
\begin{equation}
t_1 = -\frac{v_0 - v_1}{a} = \frac{-35\frac{m}{s} - 5\frac{m}{s}}{-0.6\frac{m}{s^2}} = 50s
\end{equation}
\begin{equation}
t_2 = \frac{s_2}{v_1} = \frac{1300m}{5\frac{m}{s}} = 260s
\end{equation}
\begin{equation}
t_3 = \frac{v_0 - v_1}{a_2} = \frac{30\frac{m}{s}}{0.5\frac{m}{s^2}} = 60s
\end{equation}
\begin{equation}
t_{tot} = t_1 + t_2 + t_3 = 370s
\end{equation}
\begin{equation}
s_1 = \frac{(5\frac{m}{s})^2 - (5\frac{m}{s})^2}{2*(-6\frac{m}{s^2}} = 1000;
\end{equation}
\begin{equation}
s_1 = \frac{(5\frac{m}{s})^2 - (35\frac{m}{s})^2}{2*(-6\frac{m}{s^2}} = 1000;
\end{equation}

\textbf{Aufgabe 31}

Geschwindigkeit während der Fahrt
\begin{equation}
v = 12\frac{m}{s}
\end{equation}

\begin{equation}
s = 300m
\end{equation}
\begin{equation}
a_1 = 1.5\frac{m}{s^2}
\end{equation}
\begin{equation}
a_2 = 1\frac{m}{s^2}
\end{equation}

Also:
\begin{equation}
t_1 = 12s 
\end{equation}
\begin{equation}
t_2 = 18s
\end{equation}
\begin{equation}
a = \frac{\delta v}{\delta t} -> v = a*t
\end{equation}
\begin{equation}
v = 1.0\frac{m}{s^2} * 12s
\end{equation}



\begin{verbatim}
s
|
|		    /--
|             /-----
|        /----
|  /-----
+-- s=v_0 * t + s_0
|
|
+----------------------- t
 			  
 			  
v    			  
|    			  
|    			  
|    			      
| v = v_0 = konstant   	      
+-------------------------    
|	    		      
|	    		      
|	    		      
+------------------------- t  

		   .
|		  ..
|		 ..
|	       ..
|	    ...
|	 ....
..........
|
+-------------------------
			
v			
|	              /---
|               /-----	
|         /-----	
|   /-----  v = v_0 + at
+---			
|
+-----------------------------t

	    		      
a     	    		      
|      	      		      
|      	      		      
| a = konstant		      
+---------------------------- 
|			      
|			      
|			      
+----------------------------a
\end{verbatim}
\item Mathematik
\label{sec-1-2-9-2}

Vektoren sind grössen welche nicht nur einen \textbf{Betrag} haben sondern auch eine \textbf{Richtung}


\textbf{Vektor Beispiele}


\begin{verbatim}
Base:



								  
							        /\`
						                  \
						                   \
						                    \
							             \
							              \
							             c \
							   	        \
							   	         \
							   		  \
							     `/|
							     /
							    /
							   /
							  /
							 / 
			      \				/ b
	----------------------o			       /  
		   a   	      /	       	       	      /	  
					     	     /

2.


				             \
					      \
					       \
					        \
					         \
					          \
					           \
						    \
						     X
						    /
						   /
						  /
						 /
					        /
					       /
					      /
					     /
					o   /
			     \  
     -------------------------  
			     /
\end{verbatim}


\textbf{Aufgabe 3}

\begin{enumerate}
\item (AM,) MD, BC, FE
\item (AF,) BM, ME, CD
\end{enumerate}


\textbf{Aufgabe 5}

Der Betrag von a und b muss positiv sein

\textbf{Aufgabe 6}

\begin{enumerate}
\item true
\item true
\item 
\end{enumerate}
\end{enumerate}

\subsubsection{2015-12-01}
\label{sec-1-2-10}
\begin{enumerate}
\item Physik
\label{sec-1-2-10-1}
\begin{enumerate}
\item 
\end{enumerate}
\begin{equation}
sin(f) = \frac{v_s}{v_b} = f = arcsin(\frac{v_s}{v_b}) = sin^{-1}(\frac{0.9}{4.5}) = 11.5°
\end{equation}


\begin{equation}
g = \frac{2s}{t^2}
\end{equation}


\begin{center}
\begin{tabular}{l}
0.33s\\
0.57m\\
\end{tabular}
\end{center}


Fallgeschwindigkeit formeln


\begin{equation}
F_G = G * \frac{m*M}{r^2}
\end{equation}

\begin{equation}
g = G*\frac{M}{r^2}
\end{equation}


\begin{verbatim}
80|
70|
60|
50|            /
40|          _/
30|        _-
20|      _-
10|  __--
 0|--__________________________
  0  .1  .2  .3  .4  .5  .6  .7
\end{verbatim}


Durchschnittliches s


\begin{equation}
s=0.1 => t = \sqrt{\frac{2s}{g}}
\end{equation}



\textbf{Aufgaben}

a48)


\begin{equation}
t = \sqrt{\frac{2*39m}{9.8\frac{m}{s^2}}} = 2.8212s
\end{equation}
\begin{equation}
v = \sqrt{2gh} = \sqrt{2 * 9.8\frac{m}{s^2} * 39m} = 27.658\frac{m}{s}
\end{equation}
\item 
\label{sec-1-2-10-2}
\item 
\label{sec-1-2-10-3}
\item 
\label{sec-1-2-10-4}
\item 
\label{sec-1-2-10-5}




\item Geschichte
\label{sec-1-2-10-6}

Quelle 1:
\begin{itemize}
\item Angst
\item jedoch ist die Gefahr auch ohne diese Warnung offensichtlich
\end{itemize}

Quelle 2:
\begin{itemize}
\item Spannungen mit den USA
\begin{itemize}
\item Vieleicht Kriegseintritt
\end{itemize}
\item Schneller Sieg
\end{itemize}

Quelle 3:
\begin{itemize}
\item Er meint, der U-Boot Krieg sei ein Krieg gegen die Menschheit
\item Schon vor 100 Jahren hat man bezug auf die \emph{perfekte demokratie} genommen
\begin{itemize}
\item Damals gab es in den USA noch nicht einmal das Frauenstimm resp Wahlrecht (1919)
\end{itemize}
\item Die BürgerInnen der USA seien die "Vorkämpfer[Innen] für die Rechte der Menschheit"
\end{itemize}
\end{enumerate}

\subsubsection{2015-12-15}
\label{sec-1-2-11}

\textbf{Wörter}

\emph{Wörter können nur in einem Satz bestimmt werden}

\begin{itemize}
\item Verben
\item Nomen
\item Pronomen
\item Adjektive
\begin{itemize}
\item 
\end{itemize}
\item Partikel
\begin{itemize}
\item Konjunktionen
\item Präpositionen
\item Adverben
\item Interjektionen
\end{itemize}
\end{itemize}

\subsubsection{2016-01-12}
\label{sec-1-2-12}
\begin{enumerate}
\item Deutsch
\label{sec-1-2-12-1}

\textbf{Wortarten}

\begin{enumerate}
\item veränderbar
\label{sec-1-2-12-1-1}
\begin{enumerate}
\item Nomen
\label{sec-1-2-12-1-1-1}
\item Pronomen
\label{sec-1-2-12-1-1-2}
\item Verben
\label{sec-1-2-12-1-1-3}
\item Adjektive
\label{sec-1-2-12-1-1-4}
\textbf{auf Nomen bezogen}
\end{enumerate}
\item nicht veränderbar
\label{sec-1-2-12-1-2}
\begin{enumerate}
\item Partikel
\label{sec-1-2-12-1-2-1}
\begin{enumerate}
\item auf Verb bezogen: Adverben
\label{sec-1-2-12-1-2-1-1}
\end{enumerate}
\end{enumerate}
\end{enumerate}
\end{enumerate}

\subsection{Hausaufgaben}
\label{sec-1-3}
\subsubsection{auf 2015-09-08}
\label{sec-1-3-1}
\begin{enumerate}
\item Physik
\label{sec-1-3-1-1}

Aufgabe 2)

Geschwindigkeit[v]: 1400m/s

Zeit[t]: \begin{equation} \frac{1.6s}{2} = 0.8s\end{equation}

Resultat Strecke[s]: \begin{equation}0.8s * 1400m/s = 1120m\end{equation}

\item Mathematik
\label{sec-1-3-1-2}

\begin{enumerate}
\item 
\end{enumerate}
2(x$^{\text{2}}$ + 4)$^{\text{2}}$ - 49(x$^{\text{2}}$ +4) + 300 = 0 | u = x$^{\text{2}}$ + 4
2u$^{\text{2}}$ - 49u + 300 = 0
\begin{equation}
u = \frac{49 +- \sqrt[]{2401 - 2400}}{4} = \frac{49 +- 1}{4}
\end{equation}


\begin{equation}
x^2 + 4 = 12 | -4
\end{equation}
\begin{equation}
x^2 = 8 | sqrt
\end{equation}
\begin{equation}
x = \sqrt[]{8}
\end{equation}

\begin{equation}
x^2 + 4 = 12.5 | -4
\end{equation}
\begin{equation}
x^2 = 8.5
\end{equation}

Aufgabe 15)

\begin{equation}
3(\frac{5}{6}x - \frac{16}{5}) + \frac{11}{3}x = 2(\frac{11}{2} + \frac{8}{3}x) - 11.6
\end{equation}

\begin{equation}
\frac{15}{6}x - \frac{48}{5} + \frac{10}{3}x = \frac{22}{2} + \frac{16}{3}x - \frac{11.6}{1} | \frac{n}{6}x; \frac{n}{10}
\end{equation}

\begin{equation}
\frac{15}{6}x - \frac{96}{10} + \frac{20}{6}x = \frac{110}{10} + \frac{32}{6}x - \frac{116}{10}
\end{equation}
\end{enumerate}
% Emacs 24.5.1 (Org mode 8.2.10)
\end{document}
