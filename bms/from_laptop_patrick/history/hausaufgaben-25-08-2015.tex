% Created 2015-08-23 So. 14:46
\documentclass[11pt]{article}
\usepackage[latin1]{inputenc}
\usepackage[T1]{fontenc}
\usepackage{fixltx2e}
\usepackage{graphicx}
\usepackage{longtable}
\usepackage{float}
\usepackage{wrapfig}
\usepackage{rotating}
\usepackage[normalem]{ulem}
\usepackage{amsmath}
\usepackage{textcomp}
\usepackage{marvosym}
\usepackage{wasysym}
\usepackage{amssymb}
\usepackage{hyperref}
\tolerance=1000
\date{\today}
\title{hausaufgaben-25-08-2015}
\hypersetup{
  pdfkeywords={},
  pdfsubject={},
  pdfcreator={Emacs 24.5.1 (Org mode 8.2.10)}}
\begin{document}

\maketitle
\tableofcontents

\section{Hausaufgaben auf 25. August 2015}
\label{sec-1}
\subsection{Aufgabe 9}
\label{sec-1-1}
Die \emph{East India Company} und das \emph{britische K�nigreich} vertraten grunds�tzlich �konomische 
Interessen. Anfangs ging es vor allem um den Handel mit verschiedenen G�ter, haupts�chlich
Rohstoffe. Mit der Entwicklung des Kapitalismus in die Phase des Monopolkapitalismus wurden
aufgund der immer h�ufiger werdenden Tendenzen der �berproduktion weitere Absatzm�rkte n�tig.
Aus diesem Grund wurde Indien und andere britische Kolonien mit billigen Industrieprodukten
aus Grossbritanien �berflutet.

(quelle: \emph{Der Imperialismus als h�chstes Stadium des Kapitalismus} von \emph{Wladimir Lenin})

\subsection{Aufgabe 10}
\label{sec-1-2}
Eine Voraussetzung war der technische und �konomische Fortschritt. Erst durch die 
Industrialisierung wurde es m�glich einerseits eine riesige Masse an waren zu produzieren
und diese dann auch g�nstig zu transportieren - �ber die ganze Welt. �konomisch wurde 
es zur notwendigkeit, da im Herkunftsland alle m�glichkeiten zur Profitmaximierung
bereits ausgesch�pft wurden. 

\subsection{Aufgabe 11}
\label{sec-1-3}
Nach der �bernahme des britischen K�nigreiches zwar eine Regierung eingesetzt welche de facto
die gesamte Macht �ber den Kontinent hatte, regional wurden jedoch die bisherigen Herrscher
(e.g. F�rstent�mer etc.) beibehalten. Diese mussten sich der britischen Kolonialregierung
beugen, da sie sonst mit Konsequenzen (u.a. milit�risch) rechnen mussten. 

Diese feudalen Strukturen wurden mit der Zeit abgeschaft und von den Briten eine B�rokratie
eingesetzt welche ab dann die Macht �bernahmen

\subsection{Aufgabe 12}
\label{sec-1-4}
Die �konomischen Strukturen wurden vollst�ndig zerst�rt. Vor der Kolonialisierung war in 
Indien die sog. \emph{Asiatische Produktionsweise (nach Marx)} vorherschend. Es wurde auf einfachste
Weise Ackerbau betrieben. Da die objektiven Bedingungen f�r dieses System Ideal war 
(i.e. sehr fruchtbarer Boden etc.) blieb die diese Produktionsweise f�r etwa 2000 Jahre
fast unver�ndert bestehen. (quelle: Kapitel 2 in \emph{Pakistan's other Story} von \emph{Lal Khan} 
ISBN: 978-93-5002-001-2)

Mit der Kolonialisierung wurde Teilweise eine Industrialisierung vorangetrieben, jedoch
nicht im grossen Stil.

Die gesellschaftlichen Strukturen wurden mehr oder weniger beibehalten. Das Kasten System
des Hinduismus spielte der Kolonialmacht in die H�nde, da es schon ein vorherschendes
Klassen-System gab welches man ebenfalls zur unterdr�ckung einsetzten konnte. Zu beachten
ist, dass erst durch kolonialisierung die Religion Hinduismus �berhaupt entstand. 
Urspr�nglich bestand der Hinduismus aus verschiedenen Religionen welche dann 
zu einer Religion zusammengef�gt wurde.
% Emacs 24.5.1 (Org mode 8.2.10)
\end{document}
