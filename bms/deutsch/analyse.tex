\documentclass[12pt,twoside,a4paper,twocolumn]{article}
\usepackage[utf8]{inputenc}
\usepackage[english,german]{babel}
\usepackage{utopia}
\usepackage[margin=1in]{geometry}
%\usepackage[parfill]{parskip} % paragraph indent: uncomment if no paragaph indent is needed
\usepackage{makeidx}
\usepackage[onehalfspacing]{setspace}
\usepackage{fancyhdr}
\usepackage{lastpage}
\usepackage{hyperref}
\renewcommand{\sffamily}{phv}

\newcommand{\titleText}{Textstellenauswahl zur Analyse}
\newcommand{\authorText}{Janes Thomas, Patrick Günthard}
\newcommand{\dateText}{\today}

\title{\titleText}
\author{\authorText}
\date{\dateText}

\pagestyle{fancy}
\fancyhf{}

\fancyhead[EL]{\titleText}
\fancyhead[OR]{\authorText}
\cfoot{\thepage \space von \pageref{LastPage}}

\begin{document}
	\maketitle
	\tableofcontents
	
	\section{Einleitung}
	
	In diesem Text werden Zitate aus dem Buch \textit{Kismet}\footnote{Jakob Arjouni, Kismet, 2001 (ISBN:978-3-257-23336-0)} von \textit{Jakob Arjouni} auf die verschiedenen Stilmittel der Satire analysiert.
	
	\section{Zitat 1, S. 149}
	
	\paragraph{Kontext:} Die Person \textit{Popeye} kommt während einer Diskussion zwischen anderen Personen, aggressiv gestimmt in einen Raum.
	
	\paragraph{Kritik:} Der Erzähler kritisiert \textit{Popeye} wegen seinem aggressiven Verhalten. Die Kritik bezieht sich auf verschiedene Aspekte seines Auftretens. Es gibt viele Übertreibungen seines Verhaltens: \textit{,,ein Kinn zum Türeneinschlagen''}, \textit{,,Pupillen, die sich bewegten, als hätten sie bei Tempo dreihundert  auf entgegenkommende weisse Elefanten zu achten''} und \textit{,,von um von einem Schulterende zum anderen zu sehen, musste ich ein bisschen den Kopf hin- und herdrehen, wie beim Tennis''}. Auch finden sich parodistische Elemente, so wird die Comicfigur \textit{Popeye} parodiert welche im Original mit Spinat stark wird\footnote{\href{https://de.wikipedia.org/wiki/Popeye}{Wikipedia: Popeye}}, in diesem Fall jedoch durch Kokain.
	
	
	\section{Zitat 3, S. 170}
	
	\paragraph{Kontext:} Der Protagonist verfolgte eine Gruppe junger Männer, in dem er mit seinem \textit{Mercedes} durch einen Flur eines Heimes fuhr und dabei die Wände der an den Flur angrenzenden Zimmer einriss. Nachdem er die Flüchtigen erfolgreich gefasst hatte, traf er auf den Heimleiter welcher aufgrund der Situation völlig aus der Fassung war. Der Protagonist bleibt ruhig und gibt als Witz vor, \textit{,,der Elektriker''} zu sein.
	
	\paragraph{Kritik:} Der Protagonist macht sich über den Heimleiter lustig, weil dieser so aus der Fassung war. Als der Heimleiter fragte, was los sei antwortete der Protagonist nur, dass er \textit{,,keine Ahnung''} habe und nur \textit{,,der Elektriker''} sei. Weiter meinte er, \textit{,,um neue Leitungen zu verlegen''}, wären die \textit{,,Maueröffnungen (\dots) ein bisschen gross''} um die Situation noch mehr verharmlost darzustellen.

	\section{Zitat 5, S. 184}
	
	\paragraph{Kontext:} Der Satz steht im Kontext einer ganzen Reihe von individuellen Eigenheiten von Menschen. So meint der Protagonist, es gebe Leute, denen würden \textit{,,karierte Anzüge''} stehen,  
	
	\paragraph{Kritik:} 
	
\end{document}