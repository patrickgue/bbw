\documentclass[12pt,twoside,a4paper]{article}
\usepackage[utf8]{inputenc}
\usepackage[english,german]{babel}
\usepackage{utopia}
\usepackage[margin=1in]{geometry}
\usepackage[parfill]{parskip} % paragraph indent: uncomment if no paragaph indent is needed
\usepackage{makeidx}
\usepackage[onehalfspacing]{setspace}
\usepackage{fancyhdr}
\usepackage{lastpage}
\usepackage{hyperref}
\renewcommand{\sffamily}{phv}

\newcommand{\titleText}{Analyse eines Propagandatexts der NSDAP von J. Goebbels}
\newcommand{\authorText}{Patrick Günthard}
\newcommand{\dateText}{\today}

\title{\titleText}
\author{\authorText}
\date{\dateText}

\pagestyle{fancy}
\fancyhf{}

\fancyhead[EL]{\titleText}
\fancyhead[OR]{\authorText}
\cfoot{\thepage \space von \pageref{LastPage}}

\begin{document}
	\maketitle
	\tableofcontents
	
	\section{Fragestellung}
		Welche Haltung hat Goebbels gegenüber der 
		\begin{itemize}
			\item Weimarer Republik
			\item und der Bevölkerung?
		\end{itemize}
		
	\section{Schlagwörter}
	\begin{itemize}
		\item (zielbewusste) Minderheit
		\item deutsches Volk
	\end{itemize}
	\section{Analyse}
	
	\textit{Goebbels} meint, der Staat\footnote{Die Weimarer Republik} sei degeneriert und würde von einer \textit{,,Herrschaft einer korrupten, innerlich morsch und faul gewordenen Mehrheit\footnote{SPD und Zentrum, anhm. d. Author}''}\footnote{aus ,,Der Nazi-Sozi''} regiert.
	
	Zum Volk meint er, es würde ihm das Bewusstsein fehlen und wenn es nicht einverstanden mit seinen Pläne wäre, \textit{,,dann pfeifen wir auf dieses Einverständnis''}. 
	
\end{document}