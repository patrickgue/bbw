\documentclass[a4paper]{article}
\usepackage[utf8]{inputenc}
\usepackage[margin=1in]{geometry}
\renewcommand{\familydefault}{phv}

\title{\textbf{M183}\\Insecure Direct Object Reference}
\author{Timo Bonomelli, Patrick Günthard}
\date{\today}

\newcommand{\idor}{\textit{Indirect Object Reference}}

\begin{document}
\maketitle
\section{About \idor}
\subsection{What is \idor?}
\idor are references to objects on a system (e.g. Database Keys) which are accessible by the user.
\subsection{Who can attack?}
Everyone who already has access to certain parts of the application but not to all data
\subsection{How does the attack work?}
The attacker manipulates the parameter that refers to a \textit{direct object}.

\section{Example}
\textit{See Presentation}
\begin{itemize}
  \item Java-Snippet
  \item Web Page
  \item Database example
\end{itemize} 
\section{Prevent attack}
\begin{itemize}
  \item Use Sessions 
  \item Authenticate on every access
\end{itemize}
\subsection{Sessions}
\begin{itemize}
\item No direct references are necessary because they can be saved on the server side.
\item Other references can be mapped
  \begin{itemize}
  \item Example: A dropdown can have own Ids which are mapped to direct object references on the server
  \end{itemize}  
\end{itemize}
\subsection{Authentication}
\begin{itemize}
\item check authentication on every \idor access
  \begin{itemize}
    \item Example: A random token can be generated and sent to the user. The user can only access the authorized data with this token
  \end{itemize}
\end{itemize}
\end{document}