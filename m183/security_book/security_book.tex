\documentclass[11pt,twoside,a4paper]{article}
\usepackage[utf8]{inputenc}
\usepackage[english]{babel}
\usepackage{utopia}
\usepackage[margin=1in]{geometry}
\usepackage[parfill]{parskip}
\usepackage{makeidx}
\usepackage[onehalfspacing]{setspace}
\usepackage{fancyhdr}
\usepackage{lastpage}
\usepackage{hyperref}
\renewcommand{\sffamily}{phv}
\renewcommand{\rmfamily}{ptm}
%\renewcommand{\familydefault}{\rmfamily}
\newcommand{\titleText}{Security Book}
\newcommand{\authorText}{Patrick Günthard}
\newcommand{\dateText}{\today}

\title{\titleText}
\author{\authorText}
\date{\dateText}

\pagestyle{fancy}
\fancyhf{}

\fancyhead[EL]{\titleText}
\fancyhead[OR]{\authorText}
\cfoot{\thepage \space von \pageref{LastPage}}

\begin{document}
	\maketitle
	\tableofcontents
        
        \section{Vulnrabilities defined by OWASP}
        \subsection{A4: Insecure Direct Object Reference}
        
        \subsubsection{Introduction}
        Insecure Direct Object Reference is a common vulnrability which exists
        in web applications. It occurs if a parameter (e.g. a GET parameter)
        references a object in the system. 

        The atacker normally has to be authorized to this system but does not
        have access to all data.
        
        \subsubsection{Example}

        A URL which looks like this:
        \texttt{http://example.net/page.php?user=\textit{myuser}} provides a
        page which shows the user data of the logged in user. One can easily
        change the parameter to show the data of another user:
        \texttt{http://example.net/page.php?user=\textit{someotheruser}}
        
        \subsubsection{How to prevent}
        
        \paragraph{Session Based}
        
        \begin{itemize}
        	\item No \textit{Direct Object Reference} has to be sent to the client, the references can be saved on the session
        	\item In the case references are needed, they can differ from the server side data (i.e. database) an can be remapped on the server
        \end{itemize}
        
        \paragraph{Authorization}
        \begin{itemize}
        	\item Every access is checked if the user is authorized to do that. Example: A random token can be created for each user which then is checked every time the user accesses the page
        \end{itemize}

		\paragraph{Advantages}
		
		\begin{tabular}{|l|p{4cm}|p{4cm}|}
			\hline
			& \textbf{Advantage} & \textbf{Disadvantage} \\\hline
			\textbf{Session Based} & Only one authorization has to be done, access data for Database etc. is saved on the server and is not accessible by the attacker & A session uses a lot of memory for each user. For applications with a high number of users, a session for each client is not possible i.e. a non-session solution has to be implemented \\\hline
			\textbf{Authorization} & No Session is needed i.e. less memory is used and more users can access the application & Authorization is needed every time the user accesses data which is more complex to implement\\\hline
		\end{tabular}
		
		\section{Symetrical encryption}
\end{document}
