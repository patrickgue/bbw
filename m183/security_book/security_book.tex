\documentclass[11pt,twoside,a4paper]{article}
\usepackage[utf8]{inputenc}
\usepackage[english]{babel}
\usepackage{utopia}
\usepackage[margin=1in]{geometry}
\usepackage[parfill]{parskip}
\usepackage{makeidx}
\usepackage[onehalfspacing]{setspace}
\usepackage{fancyhdr}
\usepackage{lastpage}
\usepackage{hyperref}
\renewcommand{\sffamily}{phv}

\newcommand{\titleText}{Security Book}
\newcommand{\authorText}{Patrick Günthard}
\newcommand{\dateText}{\today}

\title{\titleText}
\author{\authorText}
\date{\dateText}

\pagestyle{fancy}
\fancyhf{}

\fancyhead[EL]{\titleText}
\fancyhead[OR]{\authorText}
\cfoot{\thepage \space von \pageref{LastPage}}

\begin{document}
	\maketitle
	\tableofcontents
        
        \section{Vulnrabilities defined by OWASP}
        \subsection{A4: Insecure Direct Object Reference}
        
        \paragraph{Introduction}
        Insecure Direct Object Reference is a common vulnrability which exists
        in web applications. It occurs if a parameter (e.g. a GET parameter)
        references a object in the system. 

        The atacker normally has to be authorized to this system but does not
        have access to all data.
        
        \paragraph{Example}

        A URL which looks like this:
        \texttt{http://example.net/page.php?user=\textbf{myuser}} provides a
        page which shows the user data of the logged in user. One can easily
        change the parameter to show the data of another user:
        \texttt{http://example.net/page.php?user=\textbf{someotheruser}}

\end{document}
