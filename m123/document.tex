\documentclass[12pt,twoside,a4paper]{article}
\usepackage[utf8]{inputenc}
\usepackage[english,german]{babel}
\usepackage{utopia}
\usepackage[margin=1in]{geometry}
%\usepackage[parfill]{parskip} % paragraph indent: uncomment if no paragaph indent is needed
\usepackage{makeidx}
\usepackage[onehalfspacing]{setspace}
\usepackage{fancyhdr}
\usepackage{lastpage}
\usepackage{hyperref}
\renewcommand{\sffamily}{phv}

\newcommand{\titleText}{Terminalserver}
\newcommand{\authorText}{Patrick Günthard}
\newcommand{\dateText}{\today}

\title{\titleText}
\author{\authorText}
\date{\dateText}

\pagestyle{fancy}
\fancyhf{}

\fancyhead[EL]{\titleText}
\fancyhead[OR]{\authorText}
\cfoot{\thepage \space von \pageref{LastPage}}

\begin{document}
	\maketitle
	\tableofcontents
	
	\section{Wozu dient's?}
	
	Terminalserver dienen dazu z.B. Anwendungen auf einem zentralen Server laufen zu lassen auf welche Clients über das Netzwerk zugreifen können.
	
	\section{Wie entstand's?}
	
	Das Konzept wurde ursprünglich für die Grossrechner der 1970er und 1980er Jahre. Da es zu teuer gewesen wäre, jedem Arbeitsplatz einen eigener Rechner zur Verfügung zu stellen, wurden sogenannte Thin-Clients installiert welche sich mit . entworfen bei welchen mehrere Thin-Clients mit dem Rechner verbunden war. 
	
	\section{Wie funktioniert's?}
	
	Auf einem zentralen Server läuft ein Server auf welchen ein Client sich verbindet und 
	
	\section{Lösungen}
	\begin{tabular}{lp{12cm}}
		SSH & \textit{Secure Shell} ist ein Protokoll welches eine verschlüsselte Übertragung zulässt. Es wird hauptsächlich auf UNIX eingesetzt.\\
	\end{tabular}
	
	
	
\end{document}