\documentclass[11pt,twoside,a4paper]{article}
\usepackage[utf8]{inputenc}
\usepackage[english,german]{babel}
\usepackage{utopia}
\usepackage[margin=1in]{geometry}
\usepackage[parfill]{parskip}
\usepackage{makeidx}
\usepackage[onehalfspacing]{setspace}
\usepackage{fancyhdr}
\usepackage{lastpage}
\usepackage{hyperref}
\renewcommand{\sffamily}{phv}

\newcommand{\titleText}{Dokumentation Modulprüfung M123 an der Berufsbildungsschule Winterthur}
\newcommand{\authorText}{J. Thomas, P. Günthard}
\newcommand{\dateText}{\today}

\title{\titleText}
\author{\authorText}
\date{\dateText}

\pagestyle{fancy}
\fancyhf{}

\fancyhead[EL]{\titleText}
\fancyhead[OR]{\authorText}
\cfoot{\thepage \space von \pageref{LastPage}}

\begin{document}
	\maketitle
	\tableofcontents
	\section{Analyse der Vorgabe}
	

	\section{Setup vom ftp server}
        In einem eingeloggten GNU/Linux System kann ein ftp-Server mit folgendem Befehl installiert werden:

        \texttt{# apt-get install proftpd}

        Die Konfiguration von dieser Software befindet sich bei dieser distribution standardmässig im File \texttt{/etc/proftpd/proftpd.conf}.
	
\end{document}
